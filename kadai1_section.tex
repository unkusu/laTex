
\section{課題1}
\subsection{数値計算の条件}
使用した方程式は以下の通りである:
\[
\frac{\partial u}{\partial t} = \varepsilon^2 \frac{\partial^2 u}{\partial x^2} - (u-a)(u^2 - 1)
\]
ここで、$\varepsilon^2 = 0.5$とした。

初期条件は以下のように、ランダムな微小ノイズとした:
\[
u(x, 0) = 0.01 \times \text{randn}(x)
\]
また、境界条件は周期($u(0) = u(L)$)とした。

\subsection{結果と考察}
パラメータ$a$を $-0.5, 0, 0.5$ に変化させて、それぞれについて時間発展を計算した。
図\ref{fig:random_ic_evolution}には、各$a$における$u(x,t)$の時空間図を示す。

\begin{figure}[H]
  \centering
  \includegraphics[width=0.85\linewidth]{tex_gif_figures/kadai1_spacetime_placeholder.png}
  \caption{各$a$における$u(x,t)$の時空間図(例示)}
  \label{fig:random_ic_evolution}
\end{figure}

図より、初期に存在した多数のランダムな小ドメインが時間とともに粗大化し、最終的には系全体がいずれかの安定相($u = \pm 1$)へ収束する様子が確認できる。

\begin{itemize}
  \item $a = 0$:ポテンシャルが対称なため、ドメインの勝敗は初期平均により決まり、界面の縮小(粗大化)によって収束する。
  \item $a > 0$:$u = -1$側が安定なため、$+1$ドメインが押し込まれて最終的に$-1$相が支配的となる。
  \item $a < 0$:$+1$相の方が安定であり、$-1$ドメインが消滅する。
\end{itemize}

これは反応項
\[
f(u) = (u - a)(u^2 - 1)
\]
によって与えられるポテンシャル井戸の深さの非対称性に起因しており、浅い井戸の相がより安定な相に飲み込まれていくという傾向を示す。

なお、課題2で見られたような明確な界面速度は観察されにくい。これはランダム初期条件により多数の界面が存在し、それらの駆動が打ち消し合って平均的な速度が小さくなるためである。

\subsection{結論}
本課題では、周期境界条件下でランダムな初期条件を与えたAllen-Cahn方程式の数値計算を行い、ドメインの粗大化と最終状態の偏りを観察した。
ポテンシャルの非対称性により、初期のランダム分布であっても優勢な相が明確に現れることを確認した。
また、界面が互いに干渉しあうことで、課題2とは異なり明確な界面速度が得られにくい点も特徴として観察された。
