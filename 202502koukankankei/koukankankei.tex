\documentclass{article}
% \setlength{\textwidth}{46zw}
% \setlength{\textheight}{78zh}
%\setlength{\textwidth}{20zw}
%\setlength{\textheight}{30zh}
\addtolength{\hoffset}{-12.77pt}
\addtolength{\headsep}{-13pt}
\addtolength{\headheight}{-20pt}
\addtolength{\voffset}{-29.77pt}
\usepackage{luatexja}
\usepackage[dvipdfmx]{graphicx}
\usepackage{amsmath}
\usepackage{multirow}
\usepackage{float}
\usepackage{url}
\usepackage{bm}
\usepackage{mathrsfs}
\usepackage{amsmath, amssymb}
\usepackage{type1cm}
\usepackage[dvipdfmx]{color}
\usepackage{here}
\usepackage{physics}
\usepackage{color}
\usepackage{simpler-wick}


\def\theequation{\arabic{equation}}
\renewcommand{\baselinestretch}{1}
\title{交換関係確かめ定式化}
\author{先進理工学研究科電気情報生命工学専攻\\学籍番号5324E085 \\藤原大地\\}
\date{2024年2月3日} 
\begin{document}
\maketitle

鉛直方向にRashba外場を印加した系で,スピン多重度が

\section{今回使用した関係式}
問題に取り掛かる前に,交換関係について復習する.

スピン演算子どうしの交換関係は引数が等しい場合,次の形で書き表される.
\begin{equation}
\begin{split}
\label{}
[\hat{S_i},\hat{S_j}]=i\hbar \epsilon_{ijk} \hat{S_k}
\end{split}
\end{equation}
ただし,$\epsilon_{ijk}$は$Levi-Civita記号$と呼ばれる三次元の完全反対称テンソルであり,次のように符号を返す.
\begin{equation}  \label{eq: cases f}
    \epsilon_{ijk}=
        \begin{cases}
            +1   &   偶置換  \\
            -1        &  奇置換    \\
            0        &   i,j,kのうちふたつ以上が同じとき    \\
        \end{cases}
    \end{equation}
引数がことなる場合,スピン演算子どうしの交換関係は0である.
また,スピン演算子と運動量演算子の交換関係は引数が異なるために0である.


\section{$\hat{\mathcal{H}}$とスピン二乗演算子の交換関係}
ここから,本題に入る
本来は$[\hat{\mathcal{H}}, \hat{S}^2]$のように全ハミルトニアンに対して交換関係を考えるべきだが、$\hat{\mathcal{H}}=\hat{\mathcal{H}}_0+\hat{\mathcal{H}}_r$のうち$\hat{\mathcal{H}}_0$にはスピンの演算子が含まれていないので、
\begin{align}
[\hat{\mathcal{H}}, \hat{S}^2]
    %
    &=[\hat{\mathcal{H}}_0+\hat{\mathcal{H}}_r, \hat{S}^2]
    \nonumber\\
    %
    &=[\hat{\mathcal{H}}_0, \hat{S}^2]+[\hat{\mathcal{H}}_r, \hat{S}^2]
    \nonumber\\
    %
    &=0+[\hat{\mathcal{H}}_r, \hat{S}^2]
    \nonumber\\
    %
    &=[\hat{\mathcal{H}}_r, \hat{S}^2]
\end{align}
となる。したがって、$[\hat{\mathcal{H}}_r, \hat{S}^2]$の値だけ考えればよい。これらの二つの演算子をあらわに書き表そう.
\begin{equation}
\begin{split}
\label{}
\hat{\mathcal{H}_r}(\tau_1,\dots,\tau_N)&=\sum_{i}^{N} \hat{\boldsymbol{s}}{(\sigma_i)} \cdot \left(\boldsymbol{\Xi} \times \hat{\boldsymbol{P}}_{(\boldsymbol{r_i})}\right)\\
&= \Xi_z \sum_{i=1}^{N} \left( -{\hat{S_x}}(\sigma_i) {\hat{P}_y}{(\boldsymbol{r_i})} + {\hat{S_y}}(\sigma_i) {\hat{P}_x}{(\boldsymbol{r_i})} \right)\\
\end{split}
\end{equation}


\begin{equation}
\begin{split}
\label{}
\hat{S^2}(\tau_1,\dots,\tau_N)&=\sum_{\omega}^{x,y,z}\left(\sum_{i}^{N} \hat{S}_{\omega}\right)^2\\
 &=\sum_{\omega}^{x,y,z}\sum_{j}^{N}\sum_{l}^{N}  \hat{S_{\omega}}{(\sigma_j)} \hat{S_{\omega}}{(\sigma_l)}\\
\end{split}
\end{equation}

\begin{equation}
\begin{split}
\label{manko}
[\hat{\mathcal{H}_r} \hat{S^2} ] =& \Xi_z  \sum_{i}^{N}\left( -{\hat{S_x}}(\sigma_i) {\hat{P}_y}{(\boldsymbol{r_i})} + {\hat{S_y}}(\sigma_i) {\hat{P}_x}{(\boldsymbol{r_i})} \right) \sum_{\omega}^{x,y,z}\sum_{j}^{N}\sum_{l}^{N} \hat{S_{\omega}}{(\sigma_j)} \hat{S_{\omega}}{(\sigma_l)}\\
&- \sum_{\omega}^{x,y,z}\sum_{j}^{N}\sum_{l}^{N} \hat{S_{\omega}}{(\sigma_j)} \hat{S_{\omega}}{(\sigma_l)} \Xi_z  \sum_{i}^{N}\left( -{\hat{S_x}}(\sigma_i) {\hat{P}_y}{(\boldsymbol{r_i})} + {\hat{S_y}}(\sigma_i) {\hat{P}_x}{(\boldsymbol{r_i})} \right)\\
=& - \sum_{\omega}^{x,y,z} \sum_{i}^{N}  \sum_{j}^{N} \sum_{l}^{N}  {\hat{P}_y}{(\boldsymbol{r_i})} \left( {\hat{S_x}}(\sigma_i) \hat{S_{\omega}}{(\sigma_j)} \hat{S_{\omega}}{(\sigma_l)} -  \hat{S_{\omega}}{(\sigma_j)} \hat{S_{\omega}}{(\sigma_l)} {\hat{S_x}}(\sigma_i)\right)\\
&+ \sum_{\omega}^{x,y,z} \sum_{i}^{N} \sum_{j}^{N} \sum_{l}^{N}  {\hat{P}_x}{(\boldsymbol{r_i})}  \left( {\hat{S_y}}(\sigma_i) \hat{S_{\omega}}{(\sigma_j)} \hat{S_{\omega}}{(\sigma_l)} - \hat{S_{\omega}}{(\sigma_j)} \hat{S_{\omega}}{(\sigma_l)  {\hat{S_y}}(\sigma_i)}\right)\\
\end{split}
\end{equation}

一電子系,N電子系で交換関係を確かめる.
\subsection{一電子系の交換関係}
\begin{equation}
\begin{split}
\label{ichidensi}
[\hat{\mathcal{H}_r} \hat{S^2} ] =&- \sum_{\omega}^{x,y,z} {\hat{P}_y}{(\boldsymbol{r})} \left( {\hat{S_x}}(\sigma) \hat{S_{\omega}}{(\sigma)} \hat{S_{\omega}}{(\sigma)} -  \hat{S_{\omega}}{(\sigma)} \hat{S_{\omega}}{(\sigma)} {\hat{S_x}}(\sigma)\right)\\
&+ \sum_{\omega}^{x,y,z}  {\hat{P}_x}{(\boldsymbol{r_i})}  \left( {\hat{S_y}}(\sigma) \hat{S_{\omega}}{(\sigma)} \hat{S_{\omega}}{(\sigma)} - \hat{S_{\omega}}{(\sigma)} \hat{S_{\omega}}{(\sigma)  {\hat{S_y}}(\sigma)}\right)\\
\end{split}
\end{equation}

式\eqref{ichidensi}の第一項を分解して,それぞれの項を整理する..

\begin{equation}
\begin{split}
\label{}
&\sum_{\omega}^{x,y,z} {\hat{P}_y}{(\boldsymbol{r})} \left( {\hat{S_x}}(\sigma) \hat{S_{\omega}}{(\sigma)} \hat{S_{\omega}}{(\sigma)} -  \hat{S_{\omega}}{(\sigma)} \hat{S_{\omega}}{(\sigma)} {\hat{S_x}}(\sigma)\right)\\
=& {\hat{P}_y}{(\boldsymbol{r})} \left( {\hat{S_x}}(\sigma) \hat{S_{x}}{(\sigma)} \hat{S_{x}}{(\sigma)} -  \hat{S_{x}}{(\sigma)} \hat{S_{x}}{(\sigma)} {\hat{S_x}}(\sigma)\right)\\
+&{\hat{P}_y}{(\boldsymbol{r})} \left( {\hat{S_x}}(\sigma) \hat{S_{y}}{(\sigma)} \hat{S_{y}}{(\sigma)} -  \hat{S_{y}}{(\sigma)} \hat{S_{y}}{(\sigma)} {\hat{S_x}}(\sigma)\right)\\
+&{\hat{P}_y}{(\boldsymbol{r})} \left( {\hat{S_x}}(\sigma) \hat{S_{z}}{(\sigma)} \hat{S_{z}}{(\sigma)} -  \hat{S_{z}}{(\sigma)} \hat{S_{z}}{(\sigma)} {\hat{S_x}}(\sigma)\right)\\
\end{split}
\end{equation}
第一項は
\begin{equation}
\begin{split}
\label{}
&{\hat{S_x}}(\sigma) \hat{S_{x}}{(\sigma)} \hat{S_{x}}{(\sigma)} -  \hat{S_{x}}{(\sigma)} \hat{S_{x}}{(\sigma)} {\hat{S_x}}(\sigma)=0 \\
\end{split}
\end{equation}
第二項は
\begin{equation}
\begin{split}
\label{}
&{\hat{S_x}}(\sigma) \hat{S_{y}}{(\sigma)} \hat{S_{y}}{(\sigma)} -  \hat{S_{y}}{(\sigma)} \hat{S_{y}}{(\sigma)} {\hat{S_x}}(\sigma) \\
=&\left({\hat{S_y}}(\sigma) \hat{S_{x}}{(\sigma)} \hat{S_{y}}{(\sigma)}+i\hbar \hat{S_z}{(\sigma)}\hat{S_{y}}{(\sigma)}\right) - \left( \hat{S_{y}}{(\sigma)} \hat{S_{x}}{(\sigma)} {\hat{S_y}}(\sigma) -i\hbar  \hat{S_{y}}{(\sigma)} \hat{S_{z}}{(\sigma)} \right)\\
=& i\hbar \hat{S_z}{(\sigma)}\hat{S_{y}}{(\sigma)} +i\hbar \hat{S_y}{(\sigma)}\hat{S_{z}}{(\sigma)} \\
\end{split}
\end{equation}
第三項は
\begin{equation}
\begin{split}
\label{}
&{\hat{S_x}}(\sigma) \hat{S_{z}}{(\sigma)} \hat{S_{z}}{(\sigma)} -  \hat{S_{z}}{(\sigma)} \hat{S_{z}}{(\sigma)} {\hat{S_x}}(\sigma) \\
=&\left({\hat{S_z}}(\sigma) \hat{S_{x}}{(\sigma)} \hat{S_{z}}{(\sigma)}-i\hbar \hat{S_z}{(\sigma)}\hat{S_{y}}{(\sigma)}\right) - \left( \hat{S_{z}}{(\sigma)} \hat{S_{x}}{(\sigma)} {\hat{S_z}}(\sigma) +i\hbar  \hat{S_{y}}{(\sigma)} \hat{S_{z}}{(\sigma)} \right)\\
=& -i\hbar \hat{S_z}{(\sigma)}\hat{S_{y}}{(\sigma)} -i\hbar \hat{S_y}{(\sigma)}\hat{S_{z}}{(\sigma)} \\
\end{split}
\end{equation}
これらを足し合わせた式\eqref{ichidensi}の第一項は0になることがわかった.この結果は式\eqref{ichidensi}の第二項でも同様である.
したがって,一電子系の場合交換関係が0になることが確かめ得られた.
\begin{equation}
\begin{split}
\label{}
[\hat{\mathcal{H}_r} \hat{S^2} ] =&0
\end{split}
\end{equation}


% \subsection{二電子系の交換関係}

% \begin{equation}
% \begin{split}
% \label{nidensi}
% [\hat{\mathcal{H}_r} \hat{S^2} ] =& - \sum_{\omega}^{x,y,z} \sum_{i}^{2}  \sum_{j}^{2} \sum_{l}^{2}  {\hat{P}_y}{(\boldsymbol{r_i})} \left( {\hat{S_x}}(\sigma_i) \hat{S_{\omega}}{(\sigma_j)} \hat{S_{\omega}}{(\sigma_l)} -  \hat{S_{\omega}}{(\sigma_j)} \hat{S_{\omega}}{(\sigma_l)} {\hat{S_x}}(\sigma_i)\right)\\
% &+ \sum_{\omega}^{x,y,z} \sum_{i}^{2} \sum_{j}^{2} \sum_{l}^{2}  {\hat{P}_x}{(\boldsymbol{r_i})}  \left( {\hat{S_y}}(\sigma_i) \hat{S_{\omega}}{(\sigma_j)} \hat{S_{\omega}}{(\sigma_l)} - \hat{S_{\omega}}{(\sigma_j)} \hat{S_{\omega}}{(\sigma_l)  {\hat{S_y}}(\sigma_i)}\right)\\
% &+\sum_{\omega}^{y,z} \sum_{i}^{N}  \sum_{j \neq i}^{N} \sum_{l \neq i}^{N} {\hat{P}_y}{(\boldsymbol{r_i})} \left( {\hat{S_x}}(\sigma_i) \hat{S_{\omega}}{(\sigma_j)} \hat{S_{\omega}}{(\sigma_l)} -  \hat{S_{\omega}}{(\sigma_j)} \hat{S_{\omega}}{(\sigma_l)} {\hat{S_x}}(\sigma_i)\right)\\
% \end{split}
% \end{equation}
    
% 式\eqref{nidensi}の第1項を展開して各項について整理する.

% \begin{equation}
% \begin{split}
% \label{nidensix}
% &\sum_{\omega}^{x,y,z} \sum_{i}^{2}  \sum_{j}^{2} \sum_{l}^{2}  {\hat{P}_y}{(\boldsymbol{r_i})} \left( {\hat{S_x}}(\sigma_i) \hat{S_{\omega}}{(\sigma_j)} \hat{S_{\omega}}{(\sigma_l)} -  \hat{S_{\omega}}{(\sigma_j)} \hat{S_{\omega}}{(\sigma_l)} {\hat{S_x}}(\sigma_i)\right)\\
% =&\sum_{i}^{2}  \sum_{j}^{2} \sum_{l}^{2} {\hat{P}_y}{(\boldsymbol{r_i})} \left( {\hat{S_x}}(\sigma_i) \hat{S_{x}}{(\sigma_j)} \hat{S_{x}}{(\sigma_l)} -  \hat{S_{x}}{(\sigma_j)} \hat{S_{x}}{(\sigma_l)} {\hat{S_x}}(\sigma_i)\right)\\
% +&\sum_{\omega}^{y,z} \sum_{i}^{2}  \sum_{j \neq i}^{2} \sum_{l \neq i}^{2} {\hat{P}_y}{(\boldsymbol{r_i})} \left( {\hat{S_x}}(\sigma_i) \hat{S_{\omega}}{(\sigma_j)} \hat{S_{\omega}}{(\sigma_l)} -  \hat{S_{\omega}}{(\sigma_j)} \hat{S_{\omega}}{(\sigma_l)} {\hat{S_x}}(\sigma_i)\right)\\
% +&2\sum_{\omega}^{y,z} \sum_{i}^{2}  \sum_{j \neq i}^{2}  {\hat{P}_y}{(\boldsymbol{r_i})} \left( {\hat{S_x}}(\sigma_i) \hat{S_{\omega}}{(\sigma_i)} \hat{S_{\omega}}{(\sigma_j)} -  \hat{S_{\omega}}{(\sigma_j)} \hat{S_{\omega}}{(\sigma_i)} {\hat{S_x}}(\sigma_i)\right)\\
% +& \sum_{\omega}^{y,z} \sum_{i}^{2}   {\hat{P}_y}{(\boldsymbol{r_i})} \left( {\hat{S_{x}}}(\sigma_i) \hat{S_{\omega}}{(\sigma_i)} \hat{S_{\omega}}{(\sigma_i)} -  \hat{S_{\omega}}{(\sigma_i)} \hat{S_{\omega}}{(\sigma_i)} {\hat{S_x}}(\sigma_i)\right)\\
% \end{split}
% \end{equation}

% 第一項は



% 第二項は

% 第三項は

% \begin{equation}
% \begin{split}
% \label{}
% &\sum_{\omega}^{y,z}\sum_{i=1}^{N}  i\hbar\epsilon_{x \omega c}\left(\hat{S_c}(\sigma_i) \hat{S_{\omega}}{(\sigma_i)}+ \hat{S_{\omega}}{(\sigma_i)} \hat{S_c}(\sigma_i) \right)  {\hat{P}_y}{(\boldsymbol{r_i})}\\
% =&\sum_{i=1}^{N}  i\hbar \left(\epsilon_{x y z}\hat{S_z}(\sigma_i) \hat{S_y}{(\sigma_i)}+\epsilon_{x z y}\hat{S_z}(\sigma_i) \hat{S_y}{(\sigma_i)}\right) {\hat{P}_y}{(\boldsymbol{r_i})}+ i\hbar \left(\epsilon_{x y z}\hat{S_y}(\sigma_i) \hat{S_z}{(\sigma_i)}+\epsilon_{x z y}\hat{S_y}(\sigma_i) \hat{S_z}{(\sigma_i)}\right) {\hat{P}_y}{(\boldsymbol{r_i})}\\
% =&0
% \end{split}
% \end{equation}

% 第四項は



% 同様にして式\eqref{nidensi}の第二項も0になる.




\subsection{N電子系の交換関係}
式\eqref{manko}の中で,
スピン演算子でできた項は式\eqref{BBB}の形で書けるが,これらをab間の関係,$i,j,l$間の関係によって場合分けしたものが,式\eqref{aaa}である.
\begin{equation}
\begin{split}
\label{BBB}
{\hat{S_a}}(\sigma_i) \hat{S_{b}}{(\sigma_j)} \hat{S_{b}}{(\sigma_l)} -  \hat{S_{b}}{(\sigma_j)} \hat{S_{b}}{(\sigma_l)} {\hat{S_a}}(\sigma_i)\\
\end{split}
\end{equation}


\begin{equation}
\begin{split}
\label{aaa}
[\hat{\mathcal{H}_r}, \hat{S^2} ] =& \sum_{i}^{N}  \sum_{j}^{N} \sum_{l}^{N} {\hat{P}_y}{(\boldsymbol{r_i})} \left( {\hat{S_x}}(\sigma_i) \hat{S_{x}}{(\sigma_j)} \hat{S_{x}}{(\sigma_l)} -  \hat{S_{x}}{(\sigma_j)} \hat{S_{x}}{(\sigma_l)} {\hat{S_x}}(\sigma_i)\right)\\
&+ \sum_{i}^{N}  \sum_{j}^{N} \sum_{l}^{N} {\hat{P}_x}{(\boldsymbol{r_i})} \left( {\hat{S_y}}(\sigma_i) \hat{S_{y}}{(\sigma_j)} \hat{S_{y}}{(\sigma_l)} -  \hat{S_{y}}{(\sigma_j)} \hat{S_{y}}{(\sigma_l)} {\hat{S_y}}(\sigma_i)\right)\\
&+ \sum_{\omega}^{y,z} \sum_{i}^{N}  \sum_{j \neq i}^{N} \sum_{l \neq i}^{N} {\hat{P}_y}{(\boldsymbol{r_i})} \left( {\hat{S_x}}(\sigma_i) \hat{S_{\omega}}{(\sigma_j)} \hat{S_{\omega}}{(\sigma_l)} -  \hat{S_{\omega}}{(\sigma_j)} \hat{S_{\omega}}{(\sigma_l)} {\hat{S_x}}(\sigma_i)\right)\\
&+ \sum_{\omega}^{y,z} \sum_{i}^{N}  \sum_{j \neq i}^{N} \sum_{l \neq i}^{N} {\hat{P}_x}{(\boldsymbol{r_i})} \left( {\hat{S_y}}(\sigma_i) \hat{S_{\omega}}{(\sigma_j)} \hat{S_{\omega}}{(\sigma_l)} -  \hat{S_{\omega}}{(\sigma_j)} \hat{S_{\omega}}{(\sigma_l)} {\hat{S_y}}(\sigma_i)\right)\\
&+2 \sum_{\omega}^{y,z} \sum_{i}^{N}  \sum_{j \neq i}^{N}  {\hat{P}_y}{(\boldsymbol{r_i})} \left( {\hat{S_x}}(\sigma_i) \hat{S_{\omega}}{(\sigma_i)} \hat{S_{\omega}}{(\sigma_j)} -  \hat{S_{\omega}}{(\sigma_j)} \hat{S_{\omega}}{(\sigma_i)} {\hat{S_x}}(\sigma_i)\right)\\
&+2 \sum_{\omega}^{y,z} \sum_{i}^{N}  \sum_{j \neq i}^{N}  {\hat{P}_x}{(\boldsymbol{r_i})} \left( {\hat{S_y}}(\sigma_i) \hat{S_{\omega}}{(\sigma_i)} \hat{S_{\omega}}{(\sigma_j)} -  \hat{S_{\omega}}{(\sigma_j)} \hat{S_{\omega}}{(\sigma_i)} {\hat{S_y}}(\sigma_i)\right)\\
&+ \sum_{\omega}^{y,z} \sum_{i}^{N}   {\hat{P}_y}{(\boldsymbol{r_i})} \left( {\hat{S_x}}(\sigma_i) \hat{S_{\omega}}{(\sigma_i)} \hat{S_{\omega}}{(\sigma_i)} -  \hat{S_{\omega}}{(\sigma_i)} \hat{S_{\omega}}{(\sigma_i)} {\hat{S_x}}(\sigma_i)\right)\\
&+ \sum_{\omega}^{y,z} \sum_{i}^{N}   {\hat{P}_x}{(\boldsymbol{r_i})} \left( {\hat{S_y}}(\sigma_i) \hat{S_{\omega}}{(\sigma_i)} \hat{S_{\omega}}{(\sigma_i)} -  \hat{S_{\omega}}{(\sigma_i)} \hat{S_{\omega}}{(\sigma_i)} {\hat{S_y}}(\sigma_i)\right)\\
\end{split}
\end{equation}

それぞれの項を整理していこう.
\begin{enumerate}
    \item $a=b$のとき(1~2項)
    \begin{equation}
        \begin{split}
        \label{}
        &{\hat{S_a}}(\sigma_i) \hat{S_{a}}{(\sigma_j)} \hat{S_{a}}{(\sigma_l)} -  \hat{S_{a}}{(\sigma_j)} \hat{S_{a}}{(\sigma_l)} {\hat{S_a}}(\sigma_i)\\
        &={\hat{S_a}}(\sigma_i) \hat{S_{a}}{(\sigma_j)} \hat{S_{a}}{(\sigma_l)} -  \hat{S_{a}}{(\sigma_i)} \hat{S_{a}}{(\sigma_j)} {\hat{S_a}}(\sigma_l)\\
        &=0\\    
        \end{split}
    \end{equation}
    
    \item $a\neq b$のとき
    \begin{enumerate}
        \item  $i$が$j,l$のどちらとも等しくないとき(3~4項)
        \begin{equation}
            \begin{split}
            \label{}
            &{\hat{S_a}}(\sigma_i) \hat{S_{b}}{(\sigma_j)} \hat{S_{b}}{(\sigma_l)} -  \hat{S_{b}}{(\sigma_j)} \hat{S_{b}}{(\sigma_l)} {\hat{S_a}}(\sigma_i)\\
            =&{\hat{S_a}}(\sigma_i) \hat{S_{b}}{(\sigma_j)} \hat{S_{b}}{(\sigma_l)} - {\hat{S_a}}(\sigma_i)  \hat{S_{b}}{(\sigma_j)} \hat{S_{b}}{(\sigma_l)}    \\
            =&0    
        \end{split}
        \end{equation}

        項の総和は

\begin{equation}
\begin{split}
\label{}
&\sum_{\omega}^{y,z}\sum_{i=1}^{N} \sum_{j\neq i}^{N} i\hbar \epsilon_{x \omega c}\hat{S_{\omega}}{(\sigma_j)} \hat{S_c}(\sigma_i)\\  
=&\sum_{i=1}^{N} \sum_{j\neq i}^{N} i\hbar \epsilon_{x yz}\hat{S_{y}}{(\sigma_j)} \hat{S_z}(\sigma_i){\hat{P}_y}{(\boldsymbol{r_i})} - \sum_{i=1}^{N} \sum_{j\neq i}^{N} i\hbar \epsilon_{xyz}\hat{S_{z}}{(\sigma_j)} \hat{S_y}(\sigma_i){\hat{P}_y}{(\boldsymbol{r_i})}\\ 
=&\sum_{i=1}^{N} \sum_{j\neq i}^{N} i\hbar \epsilon_{x yz}\hat{S_{y}}{(\sigma_j)} \hat{S_z}(\sigma_i){\hat{P}_y}{(\boldsymbol{r_i})} - \sum_{j=1}^{N} \sum_{i\neq j}^{N} i\hbar \epsilon_{xyz}\hat{S_{y}}{(\sigma_i)} \hat{S_z}(\sigma_j){\hat{P}_y}{(\boldsymbol{r_i})}\\ 
=&0
\end{split}
\end{equation}


\begin{equation}
\begin{split}
\label{}
&\sum_{\omega}^{x,z}\sum_{i=1}^{N} \sum_{j\neq i}^{N} i\hbar \epsilon_{y \omega c}\hat{S_{\omega}}{(\sigma_j)} \hat{S_c}(\sigma_i)\\  
=&\sum_{i=1}^{N} \sum_{j\neq i}^{N} i\hbar \epsilon_{y xz}\hat{S_{x}}{(\sigma_j)} \hat{S_z}(\sigma_i) {\hat{P}_x}{(\boldsymbol{r_i})} - \sum_{i=1}^{N} \sum_{j\neq i}^{N} i\hbar \epsilon_{yxz}\hat{S_{z}}{(\sigma_j)} \hat{S_x}(\sigma_i) {\hat{P}_x}{(\boldsymbol{r_i})}\\ 
=&\sum_{i=1}^{N} \sum_{j\neq i}^{N} i\hbar \epsilon_{yxz}\hat{S_{x}}{(\sigma_j)} \hat{S_z}(\sigma_i) {\hat{P}_x}{(\boldsymbol{r_i})} - \sum_{j=1}^{N} \sum_{i\neq j}^{N} i\hbar \epsilon_{yxz}\hat{S_{x}}{(\sigma_i)} \hat{S_z}(\sigma_j) {\hat{P}_x}{(\boldsymbol{r_i})}\\ 
=&0
\end{split}
\end{equation}

        \item  $i$が$j$,$l$のどちらか一つと等しいとき(5~6項)
        \begin{equation}
            \begin{split}
            \label{}
            &{\hat{S_a}}(\sigma_i) \hat{S_{b}}{(\sigma_i)} \hat{S_{b}}{(\sigma_k)} -  \hat{S_{b}}{(\sigma_k)} \hat{S_{b}}{(\sigma_i)} {\hat{S_a}}(\sigma_i)\\
           =& \hat{S_{b}}{(\sigma_k)} [\hat{S_a}(\sigma_i) \hat{S_{b}}{(\sigma_i)} ]\\  
           =& i\hbar \epsilon_{abc}\hat{S_{b}}{(\sigma_k)} \hat{S_c}(\sigma_i)\\  
        \end{split}
        \end{equation}
        
        項の総和は
        
        \begin{equation}
        \begin{split}
        \label{}
        &\sum_{\omega}^{y,z}\sum_{i=1}^{N} \sum_{j\neq i}^{N} i\hbar \epsilon_{x \omega c}\hat{S_{\omega}}{(\sigma_j)} \hat{S_c}(\sigma_i)\\  
        =&\sum_{i=1}^{N} \sum_{j\neq i}^{N} i\hbar \epsilon_{x yz}\hat{S_{y}}{(\sigma_j)} \hat{S_z}(\sigma_i){\hat{P}_y}{(\boldsymbol{r_i})} - \sum_{i=1}^{N} \sum_{j\neq i}^{N} i\hbar \epsilon_{xyz}\hat{S_{z}}{(\sigma_j)} \hat{S_y}(\sigma_i){\hat{P}_y}{(\boldsymbol{r_i})}\\ 
        =&\sum_{i=1}^{N} \sum_{j\neq i}^{N} i\hbar \epsilon_{x yz}\hat{S_{y}}{(\sigma_j)} \hat{S_z}(\sigma_i){\hat{P}_y}{(\boldsymbol{r_i})} - \sum_{j=1}^{N} \sum_{i\neq j}^{N} i\hbar \epsilon_{xyz}\hat{S_{y}}{(\sigma_i)} \hat{S_z}(\sigma_j){\hat{P}_y}{(\boldsymbol{r_i})}\\ 
        =&0
        \end{split}
        \end{equation}
        
        
        \begin{equation}
        \begin{split}
        \label{}
        &\sum_{\omega}^{x,z}\sum_{i=1}^{N} \sum_{j\neq i}^{N} i\hbar \epsilon_{y \omega c}\hat{S_{\omega}}{(\sigma_j)} \hat{S_c}(\sigma_i)\\  
        =&\sum_{i=1}^{N} \sum_{j\neq i}^{N} i\hbar \epsilon_{y xz}\hat{S_{x}}{(\sigma_j)} \hat{S_z}(\sigma_i) {\hat{P}_x}{(\boldsymbol{r_i})} - \sum_{i=1}^{N} \sum_{j\neq i}^{N} i\hbar \epsilon_{yxz}\hat{S_{z}}{(\sigma_j)} \hat{S_x}(\sigma_i) {\hat{P}_x}{(\boldsymbol{r_i})}\\ 
        =&\sum_{i=1}^{N} \sum_{j\neq i}^{N} i\hbar \epsilon_{yxz}\hat{S_{x}}{(\sigma_j)} \hat{S_z}(\sigma_i) {\hat{P}_x}{(\boldsymbol{r_i})} - \sum_{j=1}^{N} \sum_{i\neq j}^{N} i\hbar \epsilon_{yxz}\hat{S_{x}}{(\sigma_i)} \hat{S_z}(\sigma_j) {\hat{P}_x}{(\boldsymbol{r_i})}\\ 
        =&0
        \end{split}
        \end{equation}
        \item  $i$が$j$,$l$のどちらか一つと等しいとき (7~8項)
        \begin{equation}
            \begin{split}
            \label{}
            &{\hat{S_a}}(\sigma_i) \hat{S_{b}}{(\sigma_i)} \hat{S_{b}}{(\sigma_i)} -  \hat{S_{b}}{(\sigma_i)} \hat{S_{b}}{(\sigma_i)} {\hat{S_a}}(\sigma_i)\\
        =& i\hbar\left(\epsilon_{abc}\hat{S_c}(\sigma_i) \hat{S_{b}}{(\sigma_i)} + {\hat{S_b}}(\sigma_i) \hat{S_{a}}{(\sigma_i)} \hat{S_{b}}{(\sigma_i)} \right)- 
         i\hbar\left( \hat{S_{b}}{(\sigma_i)} \epsilon_{bac}\hat{S_c}(\sigma_i)  + {\hat{S_b}}(\sigma_i) \hat{S_{a}}{(\sigma_i)} \hat{S_{b}}{(\sigma_i)} \right)\\
        =&i\hbar\epsilon_{abc} \left(\hat{S_c}(\sigma_i) \hat{S_{b}}{(\sigma_i)}+ \hat{S_{b}}{(\sigma_i)} \hat{S_c}(\sigma_i) \right)
        \end{split}
        \end{equation}
        
        
        項の総和は
        
        \begin{equation}
        \begin{split}
        \label{}
        &\sum_{\omega}^{y,z}\sum_{i=1}^{N}  i\hbar\epsilon_{x \omega c}\left(\hat{S_c}(\sigma_i) \hat{S_{\omega}}{(\sigma_i)}+ \hat{S_{\omega}}{(\sigma_i)} \hat{S_c}(\sigma_i) \right)  {\hat{P}_y}{(\boldsymbol{r_i})}\\
        =&\sum_{i=1}^{N}  i\hbar \left(\epsilon_{x y z}\hat{S_z}(\sigma_i) \hat{S_y}{(\sigma_i)}+\epsilon_{x z y}\hat{S_z}(\sigma_i) \hat{S_y}{(\sigma_i)}\right) {\hat{P}_y}{(\boldsymbol{r_i})}+ i\hbar \left(\epsilon_{x y z}\hat{S_y}(\sigma_i) \hat{S_z}{(\sigma_i)}+\epsilon_{x z y}\hat{S_y}(\sigma_i) \hat{S_z}{(\sigma_i)}\right) {\hat{P}_y}{(\boldsymbol{r_i})}\\
        =&0
        \end{split}
        \end{equation}
        
        \begin{equation}
        \begin{split}
        \label{}
        &\sum_{\omega}^{x,z}\sum_{i=1}^{N} i\hbar\epsilon_{y \omega c}\left(\hat{S_c}(\sigma_i) \hat{S_{\omega}}{(\sigma_i)}+ \hat{S_{\omega}}{(\sigma_i)} \hat{S_c}(\sigma_i) \right){\hat{P}_x}{(\boldsymbol{r_i})}\\
        =&\sum_{i=1}^{N}  i\hbar \left(\epsilon_{y x z}\hat{S_z}(\sigma_i) \hat{S_x}{(\sigma_i)}+\epsilon_{y z x}\hat{S_x}(\sigma_i) \hat{S_z}{(\sigma_i)}\right){\hat{P}_x}{(\boldsymbol{r_i})}+ i\hbar \left(\epsilon_{ y x z}\hat{S_x}(\sigma_i) \hat{S_z}{(\sigma_i)}+\epsilon_{x z y}\hat{S_z}(\sigma_i) \hat{S_x}{(\sigma_i)}\right){\hat{P}_x}{(\boldsymbol{r_i})}\\
        =&0
        \end{split}
        \end{equation}

    \end{enumerate}
\end{enumerate}






% \begin{equation}
% \begin{split}
% \label{}
% [\hat{\mathcal{H}_r} ,\hat{S^2} ]&= - \sum_{\omega}^{x,y,z} \sum_{i}^{N}  \sum_{j}^{N} \sum_{l}^{N}  {\hat{P}_y}{(\boldsymbol{r_i})} \left( {\hat{S_x}}(\sigma_i) \hat{S_{\omega}}{(\sigma_j)} \hat{S_{\omega}}{(\sigma_l)} -  \hat{S_{\omega}}{(\sigma_j)} \hat{S_{\omega}}{(\sigma_l)} {\hat{S_x}}(\sigma_i)\right)\\
% &+ \sum_{\omega}^{x,y,z} \sum_{i}^{N} \sum_{j  }^{N} \sum_{l }^{N}  {\hat{P}_x}{(\boldsymbol{r_i})}  \left( {\hat{S_y}}(\sigma_i) \hat{S_{\omega}}{(\sigma_j)} \hat{S_{\omega}}{(\sigma_l)} - \hat{S_{\omega}}{(\sigma_j)} \hat{S_{\omega}}{(\sigma_l)  {\hat{S_y}}(\sigma_i)}\right)\\
% &= -\sum_{\omega}^{y,z}  {\hat{P}_y}{(\boldsymbol{r_i})} \left(2\sum_{i}^{N} \sum_{j \neq i}^{N}i\hbar \epsilon_{x \omega c}(\hat{S_{\omega}}{(\sigma_j)} \hat{S_c}(\sigma_i)) +\sum_{i}^{N} i\hbar\epsilon_{x \omega c} ( \hat{S_c}(\sigma_i) \hat{S_{\omega}}{(\sigma_i)}+ \hat{S_{\omega}}{(\sigma_i)} \hat{S_c}(\sigma_i)) \right) \\
% &+\sum_{\omega}^{x,z}  {\hat{P}_x}{(\boldsymbol{r_i})}   \left(2\sum_{i}^{N} \sum_{j \neq i}^{N}i\hbar \epsilon_{y \omega c}\left(\hat{S_{\omega}}{(\sigma_j)} \hat{S_c}(\sigma_i)\right) +\sum_{i}^{N} i\hbar\epsilon_{y \omega c} \left( \hat{S_c}(\sigma_i) \hat{S_{\omega}}{(\sigma_i)}+ \hat{S_{\omega}}{(\sigma_i)} \hat{S_c}(\sigma_i)\right) \right) \\
% &= - 2 {\hat{P}_y}{(\boldsymbol{r_i})} \sum_{\omega}^{y,z}\sum_{i}^{N} \sum_{j \neq i}^{N}i\hbar \epsilon_{x \omega c}\left(\hat{S_{\omega}}{(\sigma_j)} \hat{S_c}(\sigma_i)\right)+2 {\hat{P}_x}{(\boldsymbol{r_i})} \sum_{\omega}^{x,z}\sum_{i}^{N} \sum_{j \neq i}^{N}i\hbar \epsilon_{y \omega c}\left(\hat{S_{\omega}}{(\sigma_j)} \hat{S_c}(\sigma_i)\right)\\
% &= - 2 {\hat{P}_y}{(\boldsymbol{r_i})} \sum_{i}^{N} \sum_{j \neq i}^{N}i\hbar\epsilon_{x y z}\left( \hat{S_y}{(\sigma_j)} \hat{S_z}(\sigma_i)-\hat{S_z}{(\sigma_j)} \hat{S_y}(\sigma_i)\right)+2 {\hat{P}_x}{(\boldsymbol{r_i})} \sum_{\omega}^{x,z}\sum_{i}^{N} \sum_{j \neq i}^{N}i\hbar \epsilon_{y \omega c}\left(\hat{S_{\omega}}{(\sigma_j)} \hat{S_c}(\sigma_i)\right)\\
% \end{split}
% \end{equation}


\begin{equation}
\begin{split}
\label{}
[\hat{\mathcal{H}_r} ,\hat{S^2} ]&= 0
\end{split}
\end{equation}

 \end{document}