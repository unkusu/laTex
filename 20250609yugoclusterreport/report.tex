\documentclass{article}
%\setlength{\textwidth}{46zw}
%\setlength{\textheight}{78zh}
%\setlength{\textwidth}{20zw}
%\setlength{\textheight}{30zh}
\addtolength{\hoffset}{-12.77pt}
\addtolength{\headsep}{-13pt}
\addtolength{\headheight}{-20pt}
\addtolength{\voffset}{-29.77pt}

\usepackage{luatexja}
\usepackage[dvipdfmx]{graphicx}
\usepackage{amsmath}
\usepackage{multirow}
\usepackage{float}
\usepackage{url}
\usepackage{bm}
\usepackage{mathrsfs}
\usepackage{amssymb}
\usepackage{type1cm}
\usepackage[dvipdfmx]{color}
\usepackage{here}
\usepackage{physics}
\usepackage{color}
\usepackage{simpler-wick}
\title{逆分子設計における探索パラメータの影響と結果分析}
\author{}
\date{}

\begin{document}
\maketitle

\section{はじめに}
本実験では、強化学習を用いた逆分子設計手法(REINVENT)において、探索パラメータである反復回数(\texttt{Num\_iterations})および1反復あたりの生成分子数(\texttt{Num\_molecules\_per\_iteration})が探索結果および分子の溶解度に与える影響を調査した。

\section{実験設定}
以下の設定で実験を行った:
\begin{itemize}
  \item \textbf{反復回数(\texttt{Num\_iterations})}: 50
  \item \textbf{生成分子数/反復(\texttt{Num\_molecules\_per\_iteration})}: 非公開(想定として100前後)
  \item \textbf{目的関数}: 溶解度(logS)
\end{itemize}
探索では、各反復ごとに新たな分子を生成し、その予測溶解度を評価、最良のスコアを持つ分子を保持して進化を行った。

\section{結果}
反復初期はlogS \approx -3.1程度の分子から始まり、探索を重ねるごとに改良された分子が出現し、最終的にlogS \approx -1.91の分子が得られた。表\ref{tab:result}に上位5分子のSMILESと予測logSを示す。

\begin{table}[h]
  \centering
  \caption{予測logS上位5分子}
  \label{tab:result}
  \begin{tabular}{|c|l|c|}
    \hline
    順位 & SMILES & logS \\
    \hline
    1 & O=S(NC(O)(F)c1ccncc1)C(O)S1=CC=CN=C1 & -1.9081 \\
    2 & CN(C(=O)NC(N)(O)c1ccncc1)c1ccnnn1 & -1.9397 \\
    3 & NC(O)(NS(=O)C(O)S1=CC=CN=C1)c1ccncc1 & -1.9421 \\
    4 & CN(C(=N)NC(O)(O)c1cpncn1)S1=CN=CN=C1 & -1.9443 \\
    5 & CN(C(=O)NC(C)(N)c1ccncc1)c1cncnc1 & -1.9448 \\
    \hline
  \end{tabular}
\end{table}

\section{構造の傾向}
上位分子には以下のような構造的傾向が見られた:
\begin{itemize}
  \item 窒素を含む芳香族(ピリジン環など)
  \item -OH, -NH$_2$など極性基
  \item S=OやS-S結合など電子的に影響の大きい官能基
\end{itemize}
これらの構造は水溶性の向上に寄与すると考えられる。

\section{考察}
\subsection*{反復回数の影響}
logSの推移を見ると、初期段階では大きく改善し、30反復目以降はほぼ収束した。このことから、\textbf{50反復は十分な探索回数}である一方、\textbf{より高い性能を目指すには80回以上への拡張も有効}であると予想される。

\subsection*{生成分子数の影響(仮定)}
本実験では1反復あたりの生成分子数は不明であるが、もしこの数が少なければ多様性の確保が難しく、局所最適に陥る可能性が高まる。\textbf{今後は分子数を増やすことで初期多様性を向上させ、より有望な化合物探索が可能になる}と考えられる。

\section{結論}
反復回数50回の設定で、初期よりも大幅に改良された高溶解度分子が得られた。探索の深化と多様性の両立が今後の課題であり、反復数・生成数の調整が最適化性能に与える影響は大きい。

\end{document}